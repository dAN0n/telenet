% !TEX root = ../TeleNet_Zobkov_435013.tex
\section{Выводы}

В данной лабораторной работе были реализованы клиент-серверная программа терминального доступа и эхо-сервер с разработкой собственного протокола на основе TCP и UDP. Протокол был реализован на языке C++ для операционных систем Windows и Linux.

На примере данной разработки были изучены основные приемы использования протокола транспортного уровня TCP --- транспортного механизма, предоставляющего поток данных с предварительной установкой соединения. Его преимуществом является достоверность получаемых данных за счет осуществления повторного запроса данных в случае их потери, устранения дублирования при получении копий одного пакета.

Однако TCP может не подойти в некоторых ситуациях обмена по сети вследствие медленной (по сравнению с UDP) работы. Например, передавая по сети данные, требующие быстрого отклика в реальном времени, необходимо соблюдать жесткие временные рамки, с которыми протокол TCP может не справиться.

В ходе данной работы было проведено ознакомление с протоколом UDP и реализовано простейшее клиент-серверное приложение эхо-сервера. По сравнению с TCP, UDP --- более простой, основанный на сообщениях, протокол без установления соединения, однако требует дополнительного контроля доставки сообщений ввиду следующих особенностей:

\begin{itemize}
	\item Неупорядоченности --- если два сообщения отправлены последовательно, порядок их получения не может быть предугадан;
	\item Ненадежности --- когда сообщение посылается, неизвестно, достигнет ли оно своего назначения --- оно может потеряться по пути.
\end{itemize}

Поэтому, UDP наиболее часто используется требовательными ко времени приложениями, когда небольшие потери не играют большой роли. Если же требуется надежность доставки, то предпочтительнее использовать TCP.