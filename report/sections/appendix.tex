% !TEX root = ../TeleNet_Zobkov_435013.tex
\begin{append}

\section{Часть TCP}
\subsection{Создание сервера} \label{app:createserver}

\begin{lstlisting}[language=C, label=lst:createserver]
int main(int argc, char *argv[]){

    /* ... */
    /* Обработка ключей при запуске и открытие файла users.txt */

    startWSA();

    WaitForSingleObject(hMutex, INFINITE);
    for (int i = 0; i < maxThreads; i++)
        clientDesc[i].sock = INVALID_SOCKET;
    ReleaseMutex(hMutex);

    sockaddr_in server;
    listenSocket = socket(AF_INET, SOCK_STREAM, 0);

    server.sin_addr.s_addr = htonl(INADDR_ANY);
    server.sin_port        = htons(serverPort);
    server.sin_family      = AF_INET;

    if (listenSocket < 0){
        puts("Socket failed with error");
        WSACleanup();
        exit(2);
    }

    if(bind(listenSocket, (struct sockaddr *) &server, sizeof(server)) < 0){
        puts("Bind failed. Error");
        exit(3);
    }

    if(listen(listenSocket, SOCKET_ERROR) == SOCKET_ERROR){
        puts("Listen call failed. Error");
        exit(4);
    }

    HANDLE hAccept = CreateThread(NULL, 0, acceptConnections, (void *)listenSocket, 0, NULL);

    serverProcess();

    WaitForSingleObject(hAccept, INFINITE);

    WSACleanup();
    return 0;
}
\end{lstlisting}

\subsection{Подключение клиентов} \label{app:acceptconnections}

\begin{lstlisting}[language=C, label=lst:createserver]
DWORD WINAPI acceptConnections(void *listenSocket){
    SOCKET acceptSocket;
    clientDescriptor desc;
    sockaddr_in clientInfo;
    int clientInfoSize = sizeof(clientInfo);

    while(true){
        int ind = -1;

        acceptSocket = accept((SOCKET)listenSocket, (struct sockaddr*)&clientInfo, &clientInfoSize);

        if(acceptSocket == INVALID_SOCKET) break;
        else if(acceptSocket < 0){
            puts("Accept failed");
            continue;
        }

        WaitForSingleObject(hMutex, INFINITE);
        for (int i = maxThreads - 1; i >= 0; i--)
        if(clientDesc[i].sock == INVALID_SOCKET){
            ind = i;
        }
        ReleaseMutex(hMutex);

        if(ind < 0){
            char *ip  = inet_ntoa(clientInfo.sin_addr);

            sendMSG(acceptSocket, MSG_FULL);
            shutdown(acceptSocket, SD_BOTH);
            closesocket(acceptSocket);

            cout << ip << ":" << clientInfo.sin_port << " was disconnected because of server overload" << endl;
            continue;
        }

        desc.sock = acceptSocket;
        desc.ip   = inet_ntoa(clientInfo.sin_addr);
        desc.port = clientInfo.sin_port;

        cout << "Connection request received." << endl;
        cout << "New socket was created at address " << desc.ip << ":" << desc.port << endl;

        clientDesc[ind] = desc;
        clientDesc[ind].handle = CreateThread(NULL, 0, (LPTHREAD_START_ROUTINE)clientProcess, (void *)desc.sock, 0, NULL);
    }

    WaitForSingleObject(hMutex, INFINITE);
    vector<HANDLE> hClients;
    for(int i = 0; i < maxThreads; i++)
        if(clientDesc[i].sock != INVALID_SOCKET){
            hClients.push_back(clientDesc[i].handle);
            closeSocket(i);
        }
    ReleaseMutex(hMutex);

    if (!hClients.empty()){
        WaitForMultipleObjects(hClients.size(), hClients.data(), TRUE, INFINITE);
    }
    return 0;
}
\end{lstlisting}

\subsection{Структуры, используемые для работы с клиентами} \label{app:struct}

\begin{lstlisting}[language=C, label=lst:createserver]
struct clientDescriptor{
    SOCKET sock;
    HANDLE handle;
    char *ip;
    int port;
    string login;
};

struct userData{
    string login;
    string password;
    string permissions;
    string path;
    bool online;
};

clientDescriptor clientDesc[MAX_THREADS_POSSIBLE];
vector<userData> users;
\end{lstlisting}

\subsection{Поток обработки клиентских сообщений} \label{app:clientprocess}

\begin{lstlisting}[language=C, label=lst:createserver]
DWORD WINAPI clientProcess(void* socket){
    int ind;
    bool exitFlag = false;
    state state = CONNECT;
    char buffer[packetSize];
    string line;

    for (int i = 0; i < maxThreads; i++){
        if (clientDesc[i].sock == (SOCKET)socket) ind = i;
    }

    while(!exitFlag){
        string login;
        string password;
        string cmd;
        int logInd;
        size_t pos = 0;
        line = "";

        switch(state){
            case CONNECT:
                sendMSG(clientDesc[ind].sock, MSG_WELCOME);

                if(recvS(clientDesc[ind].sock, buffer, line) != 1){
                    exitFlag = true;
                    break;
                }else cout << clientDesc[ind].ip << ":" << clientDesc[ind].port << " " << line << endl;

                pos = line.find(" ");
                cmd = line.substr(0, pos);

                if(cmd == "login" || cmd == "addusr"){
                    if(line.find_first_of(" ") != string::npos)
                        line = line.substr(line.find_first_of(" "), string::npos);
                    if(line.find_first_not_of(" ") != string::npos)
                        line = line.substr(line.find_first_not_of(" "), string::npos);
                    
                    pos = line.find(" ");
                    login = line.substr(0, pos);

                    if(line.find_first_of(" ") != string::npos)
                        line = line.substr(line.find_first_of(" "), string::npos);
                    if(line.find_first_not_of(" ") != string::npos)
                        line = line.substr(line.find_first_not_of(" "), string::npos);

                    pos = line.find(" ");
                    password = line.substr(0, pos);
                }
                else sendMSG(clientDesc[ind].sock, MSG_INVALID_CMD);
                
                if(cmd == "login"){
                    if(loginCommand(login, password) == 0){
                        logInd = getUserIndex(login);

                        WaitForSingleObject(hMutex,INFINITE);
                        clientDesc[ind].login = login;
                        ReleaseMutex(hMutex);

                        state = WORK;
                    }
                    else if(loginCommand(login, password) == 1){
                        sendMSG(clientDesc[ind].sock, MSG_LOGPASS_NOT_MATCH);
                        break;
                    }
                    else{
                        sendMSG(clientDesc[ind].sock, MSG_USER_ONLINE);
                        break;                        
                    }
                }

                if(cmd == "addusr"){
                    if(addusrCommand(login, password) == 0){
                        logInd = getUserIndex(login);

                        WaitForSingleObject(hMutex,INFINITE);
                        clientDesc[ind].login = login;
                        users[logInd].online = true;
                        ReleaseMutex(hMutex);

                        state = WORK;
                    }
                    else if(addusrCommand(login, password) == 1){
                        sendMSG(clientDesc[ind].sock, MSG_USER_EXISTS);
                        break;
                    }
                    else if(addusrCommand(login, password) == 2){
                        cout << "ERROR: Can't open users.txt" << endl;
                        return 2;
                    }
                    else{
                        sendMSG(clientDesc[ind].sock, MSG_LOGPASS_MATCHING);
                    }
                }

                break;
            case WORK:
                while(true){
                    string cmd;
                    vector<string> names;

                    string prompt = users[logInd].login + " @ " + users[logInd].path + " $ \n";
                    sendMSG(clientDesc[ind].sock, prompt);
                    
                    cmd = "";
                    if(recvS(clientDesc[ind].sock, buffer, cmd) != 1){
                        exitFlag = true;
                        break;
                    }else cout << clientDesc[ind].ip << ":" << clientDesc[ind].port
                               << "|" << clientDesc[ind].login << "| " << cmd << endl;

                    if(cmd == "ls"){
                        names = lsCommand(users[logInd].path);

                        for(int i = 0; i < names.size(); i++)
                            sendMSG(clientDesc[ind].sock, names[i] + "\n");
                        names.clear();
                    }

                    else if(cmd.find("cd ") != string::npos){
                        cmd = cmd.substr(3, string::npos);
                        cmd = cmd.substr(0, cmd.find_last_not_of(" ") + 1);

                        users[logInd].path = cdCommand(clientDesc[ind].sock, cmd, users[logInd].path);
                        rewriteUserFile();
                    }

                    else if(cmd == "pwd"){
                        sendMSG(clientDesc[ind].sock, users[logInd].path + "\n");
                    }

                    else if(cmd == "who"){
                        if(users[logInd].permissions.find("w") != string::npos){
                            for(int i = 0; i < users.size(); i++){
                                string online, send;
                                if(users[i].online) online = "ONLINE";
                                send = users[i].login + "\t" + online + "\t" + users[i].path + "\n";
                                sendMSG(clientDesc[ind].sock, send);
                            }
                        }else sendMSG(clientDesc[ind].sock, MSG_NO_PERMISSIONS);
                    }

                    else if(cmd.find("chmod ") != string::npos){
                        if(users[logInd].permissions.find("c") != string::npos){
                            cmd = cmd.substr(6, string::npos);
                            cmd = cmd.substr(0, cmd.find_last_not_of(" ") + 1);

                            pos = cmd.find(" ");
                            string login = cmd.substr(0, pos);

                            if(login == "root" || login == users[logInd].login){
                                sendMSG(clientDesc[ind].sock, MSG_LOCK_PERMISSIONS);
                                continue;
                            }

                            if(chmodCommand(cmd) == 0) rewriteUserFile();
                            else sendMSG(clientDesc[ind].sock, MSG_USER_NOT_EXISTS);
                        }else sendMSG(clientDesc[ind].sock, MSG_NO_PERMISSIONS);
                    }

                    else if(cmd.find("kill ") != string::npos){
                        if(users[logInd].permissions.find("k") != string::npos){
                            cmd = cmd.substr(5, string::npos);
                            cmd = cmd.substr(0, cmd.find_last_not_of(" ") + 1);

                            if(cmd == "root" || cmd == users[logInd].login){
                                sendMSG(clientDesc[ind].sock, MSG_LOCK_KILL);
                                continue;
                            }

                            if(killCommand(cmd) != 0) sendMSG(clientDesc[ind].sock, MSG_KILL_FAILED);
                        }else sendMSG(clientDesc[ind].sock, MSG_NO_PERMISSIONS);
                    }

                    else if(cmd == "logout"){
                        break;
                    }
                    else sendMSG(clientDesc[ind].sock, MSG_INVALID_CMD);
                }

                WaitForSingleObject(hMutex,INFINITE);
                users[logInd].online = false;
                clientDesc[ind].login = "";
                ReleaseMutex(hMutex);

                state = CONNECT;

                break;
            default:
                break;
        }
    }

    closeSocket(ind);
    return 0;
}
\end{lstlisting}

\subsection{Функция чтения до перевода строки} \label{app:recvs}

\begin{lstlisting}[language=C, label=lst:createserver]
int recvS(SOCKET socket, char *buf, string &line){
    int rc = 1;
    while(rc != 0){
        rc = recv(socket, buf, 1, 0);
        if(rc <= 0) return 0;
        if(buf[0] == '\n') return 1;
        line = line + buf[0];
    }
}
\end{lstlisting}

\subsection{Функция закрытия сокета} \label{app:close}

\begin{lstlisting}[language=C, label=lst:createserver]
void closeSocket(int ind){
    WaitForSingleObject(hMutex, INFINITE);
    if(ind >= 0 && ind < maxThreads && clientDesc[ind].sock != INVALID_SOCKET){
            if(closesocket(clientDesc[ind].sock))
                puts("Closesocket failed");
            else{
                cout << clientDesc[ind].ip << ":" << clientDesc[ind].port << " was disconnected" << endl;
                clientDesc[ind].sock = INVALID_SOCKET;
            }
    }
    ReleaseMutex(hMutex);
}
\end{lstlisting}

\section{Часть UDP} 
\subsection{Инициализация сервера} \label{app:createserverudp}

\begin{lstlisting}[language=C, label=lst:createserver]
int main(int argc, char *argv[]){

    /* ... */
    /* Обработка ключей при запуске */

    for(int i = 0; i < MAX_THREADS; i++) clientsData[i].id = -1;

    pthread_mutex_init(&mutex, NULL);

    mainSocket = socket(AF_INET, SOCK_DGRAM, IPPROTO_UDP);

    if(mainSocket < 0){
        puts("Socket failed with error");
        return 1;
    }

    memset((char *) &serverAddress, 0, sizeof(serverAddress));
    serverAddress.sin_family = AF_INET;
    serverAddress.sin_addr.s_addr = htonl(INADDR_ANY);
    serverAddress.sin_port = htons(serverPort);

    if(bind(mainSocket, (struct sockaddr*) &serverAddress, sizeof(serverAddress)) < 0){
        puts("Bind failed. Error");
        return 2;
    }

    processClients();

    int status_addr;

    pthread_join(serverThread, (void**) &status_addr);
    for(int i = 0; i < MAX_THREADS; i++) {
        pthread_join(clients[i], (void**) &status_addr);  // ЧТО-ТО СТРАННОЕ
    }

    close(mainSocket);
    mainSocket = -1;

    puts("Server is stopped.\n");
    return 0;
}
\end{lstlisting}

\subsection{Подключение клиентов} \label{app:acceptconnectionsudp}

\begin{lstlisting}[language=C, label=lst:createserver]
void processClients(){
    pthread_create(&serverThread, NULL, serverProcess, NULL);

    while(serverStart){
        struct sockaddr_in clientAddress;
        int size = sizeof(struct sockaddr_in);
        char buff[BUFLEN] = "";
        bool newThread = true;
        bool exist = false;

        recvfrom(mainSocket, buff, BUFLEN, 0, (sockaddr *) &clientAddress, (socklen_t *) &size);
        int pos = findExistingClientsData(clientAddress);
        if(pos >= 0) newThread = false;
        else pos = findFreeClientsData();

        if(pos >= 0){
            clientStruct client;
            client.id = pos;
            memcpy(client.buffer, buff, sizeof(client.buffer));
            client.cl_sock = clientAddress;

                pthread_mutex_lock(&mutex);
                clientsData[pos] = client;
                pthread_mutex_unlock(&mutex);

            if(newThread && serverStart){
                cout << "Add new client " << inet_ntoa(clientAddress.sin_addr)
                     << ":" << clientAddress.sin_port << endl;

                pthread_create(&clients[pos], NULL, clientProcess, (void*) &pos);
            }
        }else{
            cout << "SERVER OVERLOAD, KILL TO FREE THREAD" << endl;
        }
    }
}
\end{lstlisting}

\subsection{Структура, используемая для работы с клиентами} \label{app:structudp}

\begin{lstlisting}[language=C, label=lst:createserver]
struct clientStruct{
    int id;
    char buffer[BUFLEN];
    sockaddr_in cl_sock;
};

struct clientStruct clientsData[MAX_THREADS];
\end{lstlisting}

\subsection{Обработка действий клиента} \label{app:clientprocessudp}

\begin{lstlisting}[language=C, label=lst:createserver]
void* clientProcess(void *args){
    int id = *((int*) args);
    string temp = clientsData[id].buffer;
    
    int numSpace = temp.find(' ', 0);
    int idPackageInt = atoi(temp.substr(0, numSpace).data()) - 1;
    
    string buffer = "";
    clock_t t1, t2;
    bool needCheck = false;

    while(clientsData[id].id >= 0){
        if(buffer != clientsData[id].buffer){
            buffer = clientsData[id].buffer;
            string answer;

            if(count(buffer.begin(), buffer.end(), ' ') > 0){
                //поиск ID пакета
                int numSpace = buffer.find(' ', 0);
                string idPackage = buffer.substr(0, numSpace);
                buffer.erase(0, numSpace + 1);

                if(needCheck){
                    t2 = clock();
                    if(buffer == "#CHECK" && ((double)(t2 - t1) / CLOCKS_PER_SEC) < 3){
                        cout << "ID:" << id << " checked" << endl;
                    }
                    else
                        cout << "ID:" << id << " has not received a message" << endl;
                    needCheck = false;
                }
                else if(atoi(idPackage.data()) - idPackageInt == 1){
                    answer = "Echo: " + buffer + "\n";

                    idPackageInt = atoi(idPackage.data());
                    send(clientsData[id].cl_sock, idPackage + " " + answer);
                    needCheck = true;
                    t1 = clock();
                }
                else{
                    if(idPackageInt - atoi(idPackage.data()) == 0) answer = SAME_ERROR;
                    else answer = MIXING_ERROR;
                    send(clientsData[id].cl_sock, answer);
                    needCheck = true;
                    t1 = clock();
                }
            }
            else{
                answer = FAILED_ERROR;
                send(clientsData[id].cl_sock, answer);
                needCheck = true;
                t1 = clock();
            }
        buffer = clientsData[id].buffer;
        }
    }
    pthread_mutex_lock(&mutex);
    clientsData[id].id = -1;
    cout << "ID:" << id << " removed from list" << endl;
    pthread_mutex_unlock(&mutex);
}
\end{lstlisting}

\subsection{Функция отправки сообщений на клиенте} \label{app:sendudp}

\begin{lstlisting}[language=C, label=lst:createserver]
int sendData(sockaddr_in server, string msg, int id_package) {
    memset(buf, 0, BUFFER_SIZE);

    string res = "";

    char buffer[8];
    sprintf(buffer, "%d", id_package);
    string id = buffer;

    res = id + " " + msg;

    int len = sizeof(server);
    sendto(clientSocket, res.data(), res.size(), 0, (struct sockaddr *) &server, len);
    string getAnswer = receiveData(server);

    if(getAnswer.empty()){
        cout << "PACKAGE " + id + " WAS LOST" << endl;
        cout << "NO CONNECTION" << endl;
    }
    else{
        int numSpace = getAnswer.find(" ", 0);
        if(numSpace < 0){
            cout << "PACKAGE " + id + " WAS LOST" << endl;
            cout << "RECEIVED: " + getAnswer << endl;
        }
        else{
            int id_getPackage = atoi(getAnswer.substr(0, numSpace).data());
            if(id_getPackage != id_package){
                cout << "PACKAGE " + id + " WAS LOST" << endl;
                cout << "RECEIVED: " + getAnswer << endl;
            }
            else{
                getAnswer.erase(0, numSpace + 1);
                cout << getAnswer << endl;
            }
        }
        res = id + " #CHECK";
        sendto(clientSocket, res.data(), strlen(res.data()), 0, (struct sockaddr *) &server, len);
    }
    return 0;
}
\end{lstlisting}

\subsection{Функция приема сообщений на клиенте} \label{app:recvudp}

\begin{lstlisting}[language=C, label=lst:createserver]
string receiveData(sockaddr_in &server) {
    memset(buf, 0, BUFFER_SIZE);

    int len = sizeof(server);

    struct timeval tv;
    tv.tv_sec = 2;
    tv.tv_usec = 0;

    fd_set rfds;
    FD_ZERO(&rfds);
    FD_SET(clientSocket, &rfds);

    int i = select(clientSocket + 1, &rfds, NULL, NULL, &tv);

    if(i > 0){
        recvfrom(clientSocket, buf, BUFFER_SIZE, 0, (struct sockaddr *) &server, &len);
        string res(buf);
        return res;
    }else{
        cout << "TIMEOUT" << endl;
        return "";
    }
    return "";
}
\end{lstlisting}


\end{append}